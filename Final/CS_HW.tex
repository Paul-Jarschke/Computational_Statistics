% Options for packages loaded elsewhere
\PassOptionsToPackage{unicode}{hyperref}
\PassOptionsToPackage{hyphens}{url}
%
\documentclass[
]{article}
\usepackage{amsmath,amssymb}
\usepackage{iftex}
\ifPDFTeX
  \usepackage[T1]{fontenc}
  \usepackage[utf8]{inputenc}
  \usepackage{textcomp} % provide euro and other symbols
\else % if luatex or xetex
  \usepackage{unicode-math} % this also loads fontspec
  \defaultfontfeatures{Scale=MatchLowercase}
  \defaultfontfeatures[\rmfamily]{Ligatures=TeX,Scale=1}
\fi
\usepackage{lmodern}
\ifPDFTeX\else
  % xetex/luatex font selection
\fi
% Use upquote if available, for straight quotes in verbatim environments
\IfFileExists{upquote.sty}{\usepackage{upquote}}{}
\IfFileExists{microtype.sty}{% use microtype if available
  \usepackage[]{microtype}
  \UseMicrotypeSet[protrusion]{basicmath} % disable protrusion for tt fonts
}{}
\makeatletter
\@ifundefined{KOMAClassName}{% if non-KOMA class
  \IfFileExists{parskip.sty}{%
    \usepackage{parskip}
  }{% else
    \setlength{\parindent}{0pt}
    \setlength{\parskip}{6pt plus 2pt minus 1pt}}
}{% if KOMA class
  \KOMAoptions{parskip=half}}
\makeatother
\usepackage{xcolor}
\usepackage[margin=1in]{geometry}
\usepackage{color}
\usepackage{fancyvrb}
\newcommand{\VerbBar}{|}
\newcommand{\VERB}{\Verb[commandchars=\\\{\}]}
\DefineVerbatimEnvironment{Highlighting}{Verbatim}{commandchars=\\\{\}}
% Add ',fontsize=\small' for more characters per line
\usepackage{framed}
\definecolor{shadecolor}{RGB}{248,248,248}
\newenvironment{Shaded}{\begin{snugshade}}{\end{snugshade}}
\newcommand{\AlertTok}[1]{\textcolor[rgb]{0.94,0.16,0.16}{#1}}
\newcommand{\AnnotationTok}[1]{\textcolor[rgb]{0.56,0.35,0.01}{\textbf{\textit{#1}}}}
\newcommand{\AttributeTok}[1]{\textcolor[rgb]{0.13,0.29,0.53}{#1}}
\newcommand{\BaseNTok}[1]{\textcolor[rgb]{0.00,0.00,0.81}{#1}}
\newcommand{\BuiltInTok}[1]{#1}
\newcommand{\CharTok}[1]{\textcolor[rgb]{0.31,0.60,0.02}{#1}}
\newcommand{\CommentTok}[1]{\textcolor[rgb]{0.56,0.35,0.01}{\textit{#1}}}
\newcommand{\CommentVarTok}[1]{\textcolor[rgb]{0.56,0.35,0.01}{\textbf{\textit{#1}}}}
\newcommand{\ConstantTok}[1]{\textcolor[rgb]{0.56,0.35,0.01}{#1}}
\newcommand{\ControlFlowTok}[1]{\textcolor[rgb]{0.13,0.29,0.53}{\textbf{#1}}}
\newcommand{\DataTypeTok}[1]{\textcolor[rgb]{0.13,0.29,0.53}{#1}}
\newcommand{\DecValTok}[1]{\textcolor[rgb]{0.00,0.00,0.81}{#1}}
\newcommand{\DocumentationTok}[1]{\textcolor[rgb]{0.56,0.35,0.01}{\textbf{\textit{#1}}}}
\newcommand{\ErrorTok}[1]{\textcolor[rgb]{0.64,0.00,0.00}{\textbf{#1}}}
\newcommand{\ExtensionTok}[1]{#1}
\newcommand{\FloatTok}[1]{\textcolor[rgb]{0.00,0.00,0.81}{#1}}
\newcommand{\FunctionTok}[1]{\textcolor[rgb]{0.13,0.29,0.53}{\textbf{#1}}}
\newcommand{\ImportTok}[1]{#1}
\newcommand{\InformationTok}[1]{\textcolor[rgb]{0.56,0.35,0.01}{\textbf{\textit{#1}}}}
\newcommand{\KeywordTok}[1]{\textcolor[rgb]{0.13,0.29,0.53}{\textbf{#1}}}
\newcommand{\NormalTok}[1]{#1}
\newcommand{\OperatorTok}[1]{\textcolor[rgb]{0.81,0.36,0.00}{\textbf{#1}}}
\newcommand{\OtherTok}[1]{\textcolor[rgb]{0.56,0.35,0.01}{#1}}
\newcommand{\PreprocessorTok}[1]{\textcolor[rgb]{0.56,0.35,0.01}{\textit{#1}}}
\newcommand{\RegionMarkerTok}[1]{#1}
\newcommand{\SpecialCharTok}[1]{\textcolor[rgb]{0.81,0.36,0.00}{\textbf{#1}}}
\newcommand{\SpecialStringTok}[1]{\textcolor[rgb]{0.31,0.60,0.02}{#1}}
\newcommand{\StringTok}[1]{\textcolor[rgb]{0.31,0.60,0.02}{#1}}
\newcommand{\VariableTok}[1]{\textcolor[rgb]{0.00,0.00,0.00}{#1}}
\newcommand{\VerbatimStringTok}[1]{\textcolor[rgb]{0.31,0.60,0.02}{#1}}
\newcommand{\WarningTok}[1]{\textcolor[rgb]{0.56,0.35,0.01}{\textbf{\textit{#1}}}}
\usepackage{graphicx}
\makeatletter
\def\maxwidth{\ifdim\Gin@nat@width>\linewidth\linewidth\else\Gin@nat@width\fi}
\def\maxheight{\ifdim\Gin@nat@height>\textheight\textheight\else\Gin@nat@height\fi}
\makeatother
% Scale images if necessary, so that they will not overflow the page
% margins by default, and it is still possible to overwrite the defaults
% using explicit options in \includegraphics[width, height, ...]{}
\setkeys{Gin}{width=\maxwidth,height=\maxheight,keepaspectratio}
% Set default figure placement to htbp
\makeatletter
\def\fps@figure{htbp}
\makeatother
\setlength{\emergencystretch}{3em} % prevent overfull lines
\providecommand{\tightlist}{%
  \setlength{\itemsep}{0pt}\setlength{\parskip}{0pt}}
\setcounter{secnumdepth}{-\maxdimen} % remove section numbering
\ifLuaTeX
  \usepackage{selnolig}  % disable illegal ligatures
\fi
\usepackage{bookmark}
\IfFileExists{xurl.sty}{\usepackage{xurl}}{} % add URL line breaks if available
\urlstyle{same}
\hypersetup{
  pdftitle={CS\_HW1},
  hidelinks,
  pdfcreator={LaTeX via pandoc}}

\title{CS\_HW1}
\author{}
\date{\vspace{-2.5em}2024-06-04}

\begin{document}
\maketitle

\subsection{R Markdown}\label{r-markdown}

This is an R Markdown document. Markdown is a simple formatting syntax
for authoring HTML, PDF, and MS Word documents. For more details on
using R Markdown see \url{http://rmarkdown.rstudio.com}.

When you click the \textbf{Knit} button a document will be generated
that includes both content as well as the output of any embedded R code
chunks within the document. You can embed an R code chunk like this:

\begin{Shaded}
\begin{Highlighting}[]
\CommentTok{\# {-}{-}{-}{-}{-}{-}{-}{-}{-}{-}{-}{-}{-}{-}{-}{-}{-}{-}{-}{-}{-}{-}{-}{-}{-}{-}{-}{-}{-}{-}{-}{-}{-}{-}{-}{-}{-}{-}{-}{-}{-}{-}{-}{-}{-}{-}{-}{-}{-}{-}{-}{-}{-}{-}{-}{-}{-}{-}{-}{-}{-}{-}}
\CommentTok{\# Computational Statistics}
\CommentTok{\# Homework 1 {-} Problem 1}
\CommentTok{\# Names: Paul Jarschke, Jan Parlesak, Leon Löppert}
\CommentTok{\# {-}{-}{-}{-}{-}{-}{-}{-}{-}{-}{-}{-}{-}{-}{-}{-}{-}{-}{-}{-}{-}{-}{-}{-}{-}{-}{-}{-}{-}{-}{-}{-}{-}{-}{-}{-}{-}{-}{-}{-}{-}{-}{-}{-}{-}{-}{-}{-}{-}{-}{-}{-}{-}{-}{-}{-}{-}{-}{-}{-}{-}{-}}


\CommentTok{\# Problem 1.1: Conjugate Gradient Algorithm {-}{-}{-}{-}}

\NormalTok{cg }\OtherTok{\textless{}{-}} \ControlFlowTok{function}\NormalTok{(A, b, x) \{}
  \CommentTok{\# Initializations:}
\NormalTok{  r }\OtherTok{\textless{}{-}}\NormalTok{ b }\SpecialCharTok{{-}}\NormalTok{ A }\SpecialCharTok{\%*\%}\NormalTok{ x  }\CommentTok{\# initial residual vector}
\NormalTok{  p }\OtherTok{\textless{}{-}}\NormalTok{ r  }\CommentTok{\# direction vector}
\NormalTok{  j }\OtherTok{\textless{}{-}} \DecValTok{0}  \CommentTok{\# iteration counter}
  
\NormalTok{  conv }\OtherTok{\textless{}{-}} \FunctionTok{c}\NormalTok{()  }\CommentTok{\# convergence criterion tracker}
\NormalTok{  err }\OtherTok{\textless{}{-}} \FunctionTok{c}\NormalTok{()  }\CommentTok{\# error tracker}
  
\NormalTok{  conv[}\DecValTok{1}\NormalTok{] }\OtherTok{\textless{}{-}} \FunctionTok{norm}\NormalTok{(r, }\StringTok{"2"}\NormalTok{)}
  
  \ControlFlowTok{if}\NormalTok{ (}\FunctionTok{all}\NormalTok{(b }\SpecialCharTok{==} \DecValTok{0}\NormalTok{)) \{}
    \CommentTok{\# if b is a zero vector, track error}
\NormalTok{    err[}\DecValTok{1}\NormalTok{] }\OtherTok{\textless{}{-}} \FunctionTok{norm}\NormalTok{(x, }\StringTok{"I"}\NormalTok{)}
\NormalTok{  \}}
  
  \CommentTok{\# Iterate until residual norm is sufficiently small}
  \CommentTok{\# or reached max number of iterations}
  \FunctionTok{cat}\NormalTok{(}\StringTok{"Starting Conguate Gradient algorithm...}\SpecialCharTok{\textbackslash{}n}\StringTok{"}\NormalTok{)}
  \ControlFlowTok{while}\NormalTok{ ((}\FunctionTok{norm}\NormalTok{(r, }\StringTok{"2"}\NormalTok{) }\SpecialCharTok{/} \FunctionTok{norm}\NormalTok{(b, }\StringTok{"2"}\NormalTok{) }\SpecialCharTok{\textgreater{}} \FloatTok{1e{-}14}\NormalTok{) }\SpecialCharTok{\&\&}\NormalTok{ j }\SpecialCharTok{\textless{}} \DecValTok{500}\NormalTok{) \{}
\NormalTok{    alpha }\OtherTok{\textless{}{-}}\NormalTok{ (}\FunctionTok{crossprod}\NormalTok{(r)) }\SpecialCharTok{/}\NormalTok{ (}\FunctionTok{t}\NormalTok{(p) }\SpecialCharTok{\%*\%}\NormalTok{ A }\SpecialCharTok{\%*\%}\NormalTok{ p)  }\CommentTok{\# step size}
\NormalTok{    x }\OtherTok{\textless{}{-}}\NormalTok{ x }\SpecialCharTok{+} \FunctionTok{c}\NormalTok{(alpha) }\SpecialCharTok{*}\NormalTok{ p  }\CommentTok{\# Update solution vector x}
\NormalTok{    rnew }\OtherTok{\textless{}{-}}\NormalTok{ r }\SpecialCharTok{{-}} \FunctionTok{c}\NormalTok{(alpha) }\SpecialCharTok{*}\NormalTok{ (A }\SpecialCharTok{\%*\%}\NormalTok{ p)  }\CommentTok{\# new residual vector}
\NormalTok{    beta }\OtherTok{\textless{}{-}}
      \FunctionTok{crossprod}\NormalTok{(rnew) }\SpecialCharTok{/} \FunctionTok{crossprod}\NormalTok{(r)  }\CommentTok{\# calculate beta coefficient}
\NormalTok{    p }\OtherTok{\textless{}{-}}\NormalTok{ rnew }\SpecialCharTok{+} \FunctionTok{c}\NormalTok{(beta) }\SpecialCharTok{*}\NormalTok{ p  }\CommentTok{\# update direction vector}
\NormalTok{    r }\OtherTok{\textless{}{-}}\NormalTok{ rnew  }\CommentTok{\# update residual vector}
\NormalTok{    j }\OtherTok{\textless{}{-}}\NormalTok{ j }\SpecialCharTok{+} \DecValTok{1}  \CommentTok{\# update iteration counter}
\NormalTok{    conv[j] }\OtherTok{\textless{}{-}} \FunctionTok{norm}\NormalTok{(r, }\StringTok{"2"}\NormalTok{) }\SpecialCharTok{/} \FunctionTok{norm}\NormalTok{(b, }\StringTok{"2"}\NormalTok{)  }\CommentTok{\# track convergence}
    
    \ControlFlowTok{if}\NormalTok{ (}\FunctionTok{all}\NormalTok{(b }\SpecialCharTok{==} \DecValTok{0}\NormalTok{)) \{}
      \CommentTok{\# if b is a zero vector, track error}
\NormalTok{      err[j] }\OtherTok{\textless{}{-}} \FunctionTok{norm}\NormalTok{(x, }\StringTok{"I"}\NormalTok{)}
\NormalTok{    \}}
    \CommentTok{\# Iteration counter:}
    \FunctionTok{cat}\NormalTok{(}\FunctionTok{sprintf}\NormalTok{(}\StringTok{\textquotesingle{}It. \%3.0f: \%20.16e}\SpecialCharTok{\textbackslash{}n}\StringTok{\textquotesingle{}}\NormalTok{, j, conv[j]), }\StringTok{"}\SpecialCharTok{\textbackslash{}n}\StringTok{"}\NormalTok{)}
\NormalTok{  \}}
  \FunctionTok{cat}\NormalTok{(}\StringTok{"...finished!}\SpecialCharTok{\textbackslash{}n}\StringTok{"}\NormalTok{)}
  \FunctionTok{cat}\NormalTok{(}\StringTok{"The solution for x is:"}\NormalTok{, x, }\StringTok{"}\SpecialCharTok{\textbackslash{}n}\StringTok{"}\NormalTok{)}
  
  \CommentTok{\# Convergence plots (Convergence criterion over iterations):}
  \FunctionTok{par}\NormalTok{(}\AttributeTok{mfrow =} \FunctionTok{c}\NormalTok{(}\DecValTok{2}\NormalTok{, }\DecValTok{1}\NormalTok{))}
  
  \FunctionTok{plot}\NormalTok{(}
    \DecValTok{1}\SpecialCharTok{:}\NormalTok{j,}
\NormalTok{    conv,}
    \AttributeTok{type =} \StringTok{"o"}\NormalTok{,}
    \AttributeTok{col =} \StringTok{"blue"}\NormalTok{,}
    \AttributeTok{pch =} \DecValTok{16}\NormalTok{,}
    \AttributeTok{cex =} \FloatTok{0.5}\NormalTok{,}
    \AttributeTok{main =} \StringTok{"Convergence Plot"}\NormalTok{,}
    \AttributeTok{xlab =} \StringTok{"Iteration"}\NormalTok{,}
    \AttributeTok{ylab =} \StringTok{"Convergence Criterion"}
\NormalTok{  )}
  \FunctionTok{lines}\NormalTok{(}\DecValTok{1}\SpecialCharTok{:}\NormalTok{j, conv, }\AttributeTok{lty =} \DecValTok{2}\NormalTok{, }\AttributeTok{col =} \StringTok{\textquotesingle{}blue\textquotesingle{}}\NormalTok{)}
  
  \FunctionTok{plot}\NormalTok{(}
    \DecValTok{1}\SpecialCharTok{:}\NormalTok{j,}
\NormalTok{    conv,}
    \AttributeTok{log =} \StringTok{"y"}\NormalTok{,}
    \AttributeTok{type =} \StringTok{"o"}\NormalTok{,}
    \AttributeTok{col =} \StringTok{"red"}\NormalTok{,}
    \AttributeTok{pch =} \DecValTok{16}\NormalTok{,}
    \AttributeTok{cex =} \FloatTok{0.5}\NormalTok{,}
    \AttributeTok{main =} \StringTok{"Convergence Plot"}\NormalTok{,}
    \AttributeTok{xlab =} \StringTok{"Iteration"}\NormalTok{,}
    \AttributeTok{ylab =} \StringTok{"Log Convergence Criterion"}
\NormalTok{  )}
  \FunctionTok{lines}\NormalTok{(}\DecValTok{1}\SpecialCharTok{:}\NormalTok{j, conv, }\AttributeTok{lty =} \DecValTok{2}\NormalTok{, }\AttributeTok{col =} \StringTok{\textquotesingle{}red\textquotesingle{}}\NormalTok{)}
  
  \CommentTok{\# Return the convergence history and the final solution vector}
  \FunctionTok{return}\NormalTok{(}\FunctionTok{list}\NormalTok{(}\AttributeTok{conv =}\NormalTok{ conv, }\AttributeTok{x =} \FunctionTok{c}\NormalTok{(x)))}
\NormalTok{\}}


\CommentTok{\# Problem 1.2: Preconditioned Conjugate Gradient Algorithm {-}{-}{-}{-}}

\NormalTok{pcg }\OtherTok{\textless{}{-}} \ControlFlowTok{function}\NormalTok{(A, b, x, M, }\AttributeTok{tol =} \FloatTok{1e{-}14}\NormalTok{) \{}
  \CommentTok{\# Initializations: {-}{-}{-}{-}}
\NormalTok{  r }\OtherTok{\textless{}{-}}\NormalTok{ b }\SpecialCharTok{{-}}\NormalTok{ A }\SpecialCharTok{\%*\%}\NormalTok{ x  }\CommentTok{\# initial residual vector}
\NormalTok{  z }\OtherTok{\textless{}{-}} \FunctionTok{solve}\NormalTok{(M, r)  }\CommentTok{\# apply preconditioner M to the residual vector}
\NormalTok{  p }\OtherTok{\textless{}{-}}\NormalTok{ z  }\CommentTok{\# direction vector}
\NormalTok{  j }\OtherTok{\textless{}{-}} \DecValTok{1}  \CommentTok{\# iteration counter}
  
\NormalTok{  conv }\OtherTok{\textless{}{-}} \FunctionTok{c}\NormalTok{()  }\CommentTok{\# convergence criterion tracker}
\NormalTok{  err }\OtherTok{\textless{}{-}} \FunctionTok{c}\NormalTok{()  }\CommentTok{\# error tracker}
  
\NormalTok{  conv[}\DecValTok{1}\NormalTok{] }\OtherTok{\textless{}{-}} \FunctionTok{norm}\NormalTok{(r, }\StringTok{"2"}\NormalTok{)}
  
  \ControlFlowTok{if}\NormalTok{ (}\FunctionTok{all}\NormalTok{(b }\SpecialCharTok{==} \DecValTok{0}\NormalTok{)) \{}
    \CommentTok{\# if b is a zero vector, track error}
\NormalTok{    err[}\DecValTok{1}\NormalTok{] }\OtherTok{\textless{}{-}} \FunctionTok{norm}\NormalTok{(x, }\StringTok{"I"}\NormalTok{)}
\NormalTok{  \}}
  
  \CommentTok{\# Iterate until residual norm is sufficiently small}
  \CommentTok{\# or reached max number of iterations}
  \FunctionTok{cat}\NormalTok{(}\StringTok{"Starting Preconditioned Conguate Gradient algorithm...}\SpecialCharTok{\textbackslash{}n}\StringTok{"}\NormalTok{)}
  
  \ControlFlowTok{while}\NormalTok{ ((}\FunctionTok{norm}\NormalTok{(r, }\StringTok{"2"}\NormalTok{) }\SpecialCharTok{/} \FunctionTok{norm}\NormalTok{(b, }\StringTok{"2"}\NormalTok{) }\SpecialCharTok{\textgreater{}}\NormalTok{ tol) }\SpecialCharTok{\&\&}\NormalTok{ j }\SpecialCharTok{\textless{}} \DecValTok{500}\NormalTok{) \{}
\NormalTok{    alpha }\OtherTok{\textless{}{-}} \FunctionTok{crossprod}\NormalTok{(z, r) }\SpecialCharTok{/}\NormalTok{ (}\FunctionTok{t}\NormalTok{(p) }\SpecialCharTok{\%*\%}\NormalTok{ A }\SpecialCharTok{\%*\%}\NormalTok{ p)  }\CommentTok{\# step size}
\NormalTok{    x }\OtherTok{\textless{}{-}}\NormalTok{ x }\SpecialCharTok{+} \FunctionTok{c}\NormalTok{(alpha) }\SpecialCharTok{*}\NormalTok{ p  }\CommentTok{\# Update solution vector x}
\NormalTok{    rnew }\OtherTok{\textless{}{-}}\NormalTok{ r }\SpecialCharTok{{-}} \FunctionTok{c}\NormalTok{(alpha) }\SpecialCharTok{*}\NormalTok{ (A }\SpecialCharTok{\%*\%}\NormalTok{ p)  }\CommentTok{\# new residual vector}
\NormalTok{    znew }\OtherTok{\textless{}{-}} \FunctionTok{solve}\NormalTok{(M, rnew)  }\CommentTok{\# update z}
\NormalTok{    beta }\OtherTok{\textless{}{-}}
      \FunctionTok{crossprod}\NormalTok{(znew, rnew) }\SpecialCharTok{/} \FunctionTok{crossprod}\NormalTok{(z, r)  }\CommentTok{\# calculate beta coefficient}
\NormalTok{    p }\OtherTok{\textless{}{-}}\NormalTok{ znew }\SpecialCharTok{+} \FunctionTok{c}\NormalTok{(beta) }\SpecialCharTok{*}\NormalTok{ p  }\CommentTok{\# update direction vector}
\NormalTok{    r }\OtherTok{\textless{}{-}}\NormalTok{ rnew  }\CommentTok{\# update residual vector}
\NormalTok{    z }\OtherTok{\textless{}{-}}\NormalTok{ znew  }\CommentTok{\# update z}
\NormalTok{    j }\OtherTok{\textless{}{-}}\NormalTok{ j }\SpecialCharTok{+} \DecValTok{1}  \CommentTok{\# update iteration counter}
    
\NormalTok{    conv[j] }\OtherTok{\textless{}{-}} \FunctionTok{norm}\NormalTok{(r, }\StringTok{"2"}\NormalTok{) }\SpecialCharTok{/} \FunctionTok{norm}\NormalTok{(b, }\StringTok{"2"}\NormalTok{)  }\CommentTok{\# track convergence}
    
    \ControlFlowTok{if}\NormalTok{ (}\FunctionTok{all}\NormalTok{(b }\SpecialCharTok{==} \DecValTok{0}\NormalTok{)) \{}
      \CommentTok{\# if b is a zero vector, track error}
\NormalTok{      err[j] }\OtherTok{\textless{}{-}} \FunctionTok{norm}\NormalTok{(x, }\StringTok{"I"}\NormalTok{)}
\NormalTok{    \}}
    \CommentTok{\# Iteration counter:}
    \FunctionTok{cat}\NormalTok{(}\FunctionTok{sprintf}\NormalTok{(}\StringTok{\textquotesingle{}It. \%3.0f: \%20.16e}\SpecialCharTok{\textbackslash{}n}\StringTok{\textquotesingle{}}\NormalTok{, j, conv[j]), }\StringTok{"}\SpecialCharTok{\textbackslash{}n}\StringTok{"}\NormalTok{)}
\NormalTok{  \}}
  \FunctionTok{cat}\NormalTok{(}\StringTok{"...finished!}\SpecialCharTok{\textbackslash{}n}\StringTok{"}\NormalTok{)}
  \FunctionTok{cat}\NormalTok{(}\StringTok{"The solution for x is:"}\NormalTok{, x, }\StringTok{"}\SpecialCharTok{\textbackslash{}n}\StringTok{"}\NormalTok{)}
  
  \CommentTok{\# Convergence plots (Convergence criterion over iterations):}
  \FunctionTok{plot}\NormalTok{(}
    \DecValTok{1}\SpecialCharTok{:}\NormalTok{j,}
\NormalTok{    conv,}
    \AttributeTok{type =} \StringTok{"o"}\NormalTok{,}
    \AttributeTok{col =} \StringTok{"blue"}\NormalTok{,}
    \AttributeTok{pch =} \DecValTok{16}\NormalTok{,}
    \AttributeTok{cex =} \FloatTok{0.5}\NormalTok{,}
    \AttributeTok{main =} \StringTok{"Convergence Plot"}\NormalTok{,}
    \AttributeTok{xlab =} \StringTok{"Iteration"}\NormalTok{,}
    \AttributeTok{ylab =} \StringTok{"Convergence Criterion"}
\NormalTok{  )}
  \FunctionTok{lines}\NormalTok{(}\DecValTok{1}\SpecialCharTok{:}\NormalTok{j, conv, }\AttributeTok{lty =} \DecValTok{2}\NormalTok{, }\AttributeTok{col =} \StringTok{"blue"}\NormalTok{)}
  
  \FunctionTok{plot}\NormalTok{(}
    \DecValTok{1}\SpecialCharTok{:}\NormalTok{j,}
\NormalTok{    conv,}
    \AttributeTok{type =} \StringTok{"o"}\NormalTok{,}
    \AttributeTok{col =} \StringTok{"red"}\NormalTok{,}
    \AttributeTok{pch =} \DecValTok{16}\NormalTok{,}
    \AttributeTok{cex =} \FloatTok{0.5}\NormalTok{,}
    \AttributeTok{main =} \StringTok{"Convergence Plot"}\NormalTok{,}
    \AttributeTok{xlab =} \StringTok{"Iteration"}\NormalTok{,}
    \AttributeTok{ylab =} \StringTok{"Log Convergence Criterion"}
\NormalTok{  )}
  \FunctionTok{lines}\NormalTok{(}\DecValTok{1}\SpecialCharTok{:}\NormalTok{j, conv, }\AttributeTok{lty =} \DecValTok{2}\NormalTok{, }\AttributeTok{col =} \StringTok{"red"}\NormalTok{)}
  
  \CommentTok{\# Return the convergence history and the final solution vector}
  \FunctionTok{return}\NormalTok{(}\FunctionTok{list}\NormalTok{(}\AttributeTok{conv =}\NormalTok{ conv, }\AttributeTok{x =} \FunctionTok{c}\NormalTok{(x)))}
\NormalTok{\}}
\end{Highlighting}
\end{Shaded}

\subsection{Including Plots}\label{including-plots}

You can also embed plots, for example:

\includegraphics{CS_HW_files/figure-latex/pressure-1.pdf}

Note that the \texttt{echo\ =\ FALSE} parameter was added to the code
chunk to prevent printing of the R code that generated the plot.

\end{document}
